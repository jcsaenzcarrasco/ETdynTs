\documentclass{elsarticle}

\biboptions{sort&compress}
\bibliographystyle{alpha}

\usepackage{amsmath,amssymb,amsfonts}
\usepackage{csquotes}
\usepackage{float}
\usepackage{subcaption}

% HASKELL SYNTAX HIGHLIGHTING - MUST BE LOADED BEFORE xcolor and listings IF THEY'RE ALSO BEING USED
\usepackage{mpshaskell}
\newcommand{\code}[1]{\haskell{#1}}
\newCodeWord{tree,list,original,reversed,x,xs,f,f',f'',y,ys,tour_x,tour_y,ex,ey,h,h',h'',e,e1,e2}
\newFunction{empty,singleton,member,reroot,splice,error,select,link,fromList',cut,path,guard,father}
\newSymbol{\insertL}{\lhd}
\newSymbol{\insertR}{\rhd}
% END OF HASKELL SYNTAX HIGHLIGHTING

\usepackage{xcolor}
\usepackage{graphicx} 
\usepackage{xspace}
\usepackage{url}

% PROOF-READING COMMANDS
\newcommand{\tcr} [1]{\textcolor{red}{#1}}
\newcommand{\tcb} [1]{\emph{\textcolor{blue}{#1}}}
\newcommand{\madd}[1]{\textcolor{Purple}{#1}}
\newcommand{\mdel}[1]{\textcolor{Yellow}{#1}}

% FORMATTING
\newcommand{\MATHSF}[1]{\ensuremath{\mathsf{#1}}\xspace}

% SYNTAX
\newcommand{\link}{\MATHSF{link}}
\newcommand{\cut}{\MATHSF{cut}}
\renewcommand{\root}{\MATHSF{root}}
\newcommand{\reroot}{\MATHSF{reroot}}
\newcommand{\conn}{\MATHSF{conn}}
\newcommand{\dyntset}{\MATHSF{dynTsET}}


\begin{document}
\title{Dynamic Trees through Euler Tours\\%
\large{A purely functional programming approach}}

\author{J.C.~Saenz-Carrasco\fnref{conacyt}\corref{cor1}}
\ead{jcsaenzcarrasco1@sheffield.ac.uk}

\author{M.~Stannett}
\ead{m.stannett@sheffield.ac.uk}

\address{Department of Computer Science,\\
Regent Court, 211 Portobello,\\
Sheffield S1 4DP, United Kingdom}

\cortext[cor1]{Corresponding author}
\fntext[conacyt]{Postgraduate research student, supported by the Consejo Nacional de Ciencia y Tecnolog\'{i}a, CONACYT (Mexican National Council for Science and Technology) under grant 411550, scholar 580617, CVU 214885.}

\begin{abstract}
In this paper we show how functional-programming features, most notably the use of lazy evaluation, can be exploited when modelling massive dynamically changing networks. Such networks are increasingly ubiquitous and important to modern commerce and society, but with growing size comes additional computational overheads, and it is important to determine whether low-complexity, preferably sub-linear, algorithms can be derived for such key tasks as finding paths between nodes, re-routing network traffic due to local outages, or distributing workloads strategically to maximize network efficiency.

We introduce a novel dynamic data structure which supports $O(\log n)$ amortised complexity for all three of the basic dynamic operations (\link, \cut and \conn), where $n$ is the number of vertices in the network. While existing functional structures mirror these time bounds in isolation, ours appears to be the first to support all three of the operations within a single structure. To illustrate how the algorithms perform in practice we have implemented them in Haskell and conducted various benchmarking experiments. The data reported here support our analysis, while also pointing to remaining inefficiencies. We discuss these in detail and suggest additional ways forward.
\end{abstract}

\begin{keyword}
Dynamic tree algorithms \sep functional programming \sep persistent data structures \sep Haskell
\end{keyword}


\maketitle



\tcr{---------- P O S I T I V E    C O M M E N T S ------------------}


\tcr{This paper introduces Euler Tour Finger Trees (ETFTs), a persistent, purely-functional tree and forest data structure that supports O(log n) complexity for all operations, including link and cut. It gives an implementation in Haskell.}

\tcr{At a high-level, the tree is represented as an Euler Tour (a list of vertex pairs describing a path through the tree), which in turn is represented by a finger tree, thereby allowing O(log n) concatenation and splitting. That allows manipulating Euler tours efficiently, and building interesting tree operations easily.}

\tcr{Most of the implementation in this paper thus builds on existing implementations of finger trees and sets. Yet the combination seems novel, and the complexity bounds of the ETFT operations follow almost trivially, which is nice.
The paper presents a functional data structure for maintaining a dynamic forest under link and cut operations. It realizes this in O(log n) time complexity per operation (amortized), using an Euler tour tree representation implemented using a combination of finger trees and sets. }

\tcr{The paper first recalls dynamic trees and their operations, and then monoids, sets implemented with balanced search trees, Euler-tour trees and finger trees. It then gives (graphical) examples of how the sets and finger trees fit together to realize dynamic trees, it recalls graphically the root, reroot, split, concatenation, and view operations of finger trees, and then walks through the implementation of root, reroot, connected, link, and cut.
Finally the paper contains experiments illustrating that the implementation meets the claimed complexity bounds and discusses related and future work.
This is an interesting piece of work.}




\section{Introduction}
\label{sec:Intro}

In this paper we show how functional programming features (for example, the use of lazy evaluation) can be exploited in the context of modelling massive dynamically changing networks. Such networks are increasingly ubiquitous and provide important value-added to modern commerce and society. With growing size, however, comes additional computational overheads, and it is important to identify low-complexity, preferably sub-linear, algorithms for such heavily utilised tasks as path-finding, re-routing and dynamic workload distribution.

Our contribution in this paper is to provide a novel functional programming data structure for network representation that allows the programming of suitable low-complexity algorithms, and we provide experimental confirmation of our performance claims. As explained in more detao; below, we work with collections of spanning trees (one for each connected component of the network), and in this context our operations show the following complexities (Figure)

\begin{center}
\begin{tabular}{||l | l | c||} 
\hline
Function & Description & Complexity \\ %[0.5ex] 
\hline\hline
\texttt{linkTree} 
  & 
  & $O(m\log n)$ \\
\texttt{cutTree} 
  &
  & $O(\log n)$ \\
\hline
\texttt{conn} 
  & 
  & $O(\log n)$ \\ 
\texttt{link} 
  & 
  & $O(m\log n)$ \\
\texttt{cut} 
  & 
  & $O(\log n)$ \\
\hline
\texttt{reroot} 
  &  
  & $O(\log n)$ \\
\texttt{root} 
  & 
  & $O(1)$ \\
\hline   
\end{tabular}
\caption{Amortised time complexity for various dynamic network-editing operations in terms of their effect on the associated forest of spanning trees. In each case, $n$ gives the number of vertices in the first (or only) tree operated upon; for those functions taking two trees as input, $m$ i the number of verticies in the second tree. The result for \link assumes that $m \leq n$ (if not, we can swap the order of the arguments before applying the algorithm).
\normalsize 
\end{center}



give first work-optimal bounds for managing dynamic trees for purely functional data structures. We do this not only for extending the application of finger trees, we offer a solution for dynamic connectivity when facing persistent data structures as well as providing a simple interface to the user. The \dyntset exposes two types, \texttt{Tree} and \texttt{Forest}, to the user. The following is a compilation of \tcb{OUR} the functions, types and times we have implemented in Haskell, on top of that done by Hinze and Paterson \cite{FTs}:

\begin{center}
\small
 \begin{tabular}{||l | l | c||} 
 \hline
 Function & Type & Time complexity \\ %[0.5ex] 
 \hline\hline
 \texttt{linkTree} & \texttt{Vertex} $\to$ \texttt{Tree} $\to$ \texttt{Vertex} $\to$ \texttt{Tree} $\to$ \texttt{Tree} & $O(m\log n)$ \\
 \texttt{cutTree} & \texttt{Vertex} $\to$ \texttt{Vertex} $\to$ \texttt{Tree} $\to$ \texttt{(Tree,Tree)} & $O(\log n)$ \\
 \hline
 \texttt{conn} & \texttt{Vertex} $\to$ \texttt{Vertex} $\to$ \texttt{Forest} $\to$ \texttt{Bool} & $O(\log n)$ \\ 
 \texttt{link} & \texttt{Vertex} $\to$ \texttt{Vertex} $\to$ \texttt{Forest} $\to$ \texttt{Forest} & $O(m\log n)$ \\
 \texttt{cut} & \texttt{Vertex} $\to$ \texttt{Vertex} $\to$ \texttt{Forest} $\to$ \texttt{Forest} & $O(\log n)$ \\
 \hline
 \texttt{reroot} & \texttt{Tree} $\to$ \texttt{Vertex} $\to$ \texttt{Tree} & $O(\log n)$ \\
 \texttt{root} & \texttt{Tree} $\to$ \texttt{Maybe(Vertex)} & $O(1)$ \\
 \hline   
\end{tabular}
where $n$ and $m$ are the corresponding sizes of the trees involved; when $linking$ is performed, then the constraint $m \leq n$ should taken into account
\end{center}
\normalsize 


In general, the number of vertices in such a network can change from one moment to the next, and as new edges are introduced and existing one deleted, the underlying topology of the network will change, with connected components merging and separating as time passes. The problem we address is one of the simplest we can ask about a network, but at the same time one of the most fundamental, namely: \emph{given two vertices, $u$ and $v$, do they belong to the same component}? The answer, of course, will depend on when we ask the question, because the system we're considering is \emph{dynamic}; new connections can be created between vertices and old ones can be deleted.

If we are to answer this question efficiently, it follows that we need to use a dynamic data type to store information about the network, its vertices, and edges. Our goal is to ensure that updates to the network will result in only local changes to the data representation; we do not want to recompute the entire representation simply because one edge has been added or removed. Our approach is first to represent connected components not as graphs but as spanning trees, and then to use a novel approach to representing these trees dyanmically. [discussion: While adding a vertex to a component and adding it to the spanning tree are equivalent tasks, breaking an edge in the spanning tree corresponds to breaking multiple edges in the underlying component.]



Given our use of spanning trees to represent connected components, this question becomes: \emph{do $u$ and $v$ belong to the same tree}? Notice that adding new links within an already connected component does not affect its existing spanning trees; although new spanning trees may become possible, no existing spanning tree loses that status. On the other hand, if we connect vertices from distinct components, joining those nodes in the associated spanning trees automatically creates a new spanning tree for the larger component created by adding the link. (The relationship between deleting an edge in a component vs. deleting it in its spanning tree is more complex--see Sect.~\ref{sec:discussion}.)  We accordingly assume that the forest can change dynamically in response to repeated applications of two basic operations: \link and \cut. 
\begin{itemize}
\item
  if vertices $u$ and $v$ belong to different trees, $\link(u,v)$ adds the edge $(u,v)$ to the forest, thereby causing the trees containing $u$ and $v$ to be joined together to form a new, larger, tree; if the vertices already belong to the same tree, the operation has no effect. Formulated in this way, the function \link, can be denoted as
\item
  if vertices $u$ and $v$ are connected by an edge, $\cut(u,v)$ removes that edge, thereby causing the tree that contains it to be split into two smaller trees; if the vertices are not connected, the operation has no effect. 
\end{itemize}

\madd{WORKING HERE}

The problem has been studied as \textit{dynamic trees} under different variants according to the application, for instance, Sleator and Tarjan defined the \textit{link-cut} tree for solving network flow problems \cite{DS-DynTs}, Henzinger and King provided \textit{Euler tour} (ET) trees for speeding up dynamic connectivity on graphs \cite{Rand-DynGs-Algos} or Frederickson describing the \textit{topology} tree for maintaining the minimum spanning tree \cite{DSs-Online-Upd-MSTs}. The former and latter trees are based on techniques called \textit{path-decomposition} and \textit{tree contraction} respectively. In this paper we are interested on Euler tour trees, a technique called \textit{linearisation}, since the original tree (of any degree) is flattened and handled it as a sequence, in other words, turning a non-linear structure into a linear one.

All the above techniques have been implemented and studied for ephemeral data structures, performing $O(\log n)$ per operation. Functional data structures, on the other hand, ease the reasoning and implementation for the cases when preserving history is needed as in self-adjusting computation \cite{DynamizingAlgos} or as in version control \cite{CVS-Demaine}. To overcome the lack of pointers, researchers have devised data structures to represent efficient sequences, most notably, \textit{finger trees} \cite{FTs}. This structure performs well, allowing updates and queries in logarithmic time. We introduce the \emph{dynamic trees through Euler-tours}, a variant of the Hinze's and Paterson finger trees, or \dyntset for short. Like the finger tree, the \dyntset edit sequences in logarithmic time while providing dynamic tree operations. The key insight is to make $O(1)$ steps to perform \textit{connected}, \textit{link}, and \textit{cut} in logarithmic time. In Section \ref{sec:TechDes} we show this for \textit{connected} and \textit{cut}, whereas \textit{link} takes $O(m \log n)$. In Section \ref{sec:Eval} we show that, in practice, different results can be obtained, such $O(1)$ per operation as long as \link is interleaved with \cut.
 \tcb{NO FURTHER explanation about the $O(1)$ performance}

\madd{MOVED THIS TO TOP -- NEEDS A BIT IF REWRITING}

The contribution of this paper is to give first work-optimal bounds for managing dynamic trees for purely functional data structures. We do this not only for extending the application of finger trees, we offer a solution for dynamic connectivity when facing persistent data structures as well as providing a simple interface to the user. The \dyntset exposes two types, \texttt{Tree} and \texttt{Forest}, to the user. The following is a compilation of \tcb{OUR} the functions, types and times we have implemented in Haskell, on top of that done by Hinze and Paterson \cite{FTs}:

\begin{center}
\small
 \begin{tabular}{||l | l | c||} 
 \hline
 Function & Type & Time complexity \\ %[0.5ex] 
 \hline\hline
 \texttt{linkTree} & \texttt{Vertex} $\to$ \texttt{Tree} $\to$ \texttt{Vertex} $\to$ \texttt{Tree} $\to$ \texttt{Tree} & $O(m\log n)$ \\
 \texttt{cutTree} & \texttt{Vertex} $\to$ \texttt{Vertex} $\to$ \texttt{Tree} $\to$ \texttt{(Tree,Tree)} & $O(\log n)$ \\
 \hline
 \texttt{conn} & \texttt{Vertex} $\to$ \texttt{Vertex} $\to$ \texttt{Forest} $\to$ \texttt{Bool} & $O(\log n)$ \\ 
 \texttt{link} & \texttt{Vertex} $\to$ \texttt{Vertex} $\to$ \texttt{Forest} $\to$ \texttt{Forest} & $O(m\log n)$ \\
 \texttt{cut} & \texttt{Vertex} $\to$ \texttt{Vertex} $\to$ \texttt{Forest} $\to$ \texttt{Forest} & $O(\log n)$ \\
 \hline
 \texttt{reroot} & \texttt{Tree} $\to$ \texttt{Vertex} $\to$ \texttt{Tree} & $O(\log n)$ \\
 \texttt{root} & \texttt{Tree} $\to$ \texttt{Maybe(Vertex)} & $O(1)$ \\
 \hline   
\end{tabular}
where $n$ and $m$ are the corresponding sizes of the trees involved; when $linking$ is performed, then the constraint $m \leq n$ should taken into account
\end{center}
\normalsize 

The first block (first two rows) contains the functions to perform dynamic tree operations. The following block (rows 4, 5 and 6) lists the functions that compute an unbounded, although finite, sequence of dynamic tree operations over the same forest $F$. Finally, last two rows are the core functions, apart from the ones provided by Hinze and Paterson work \cite{FTs}. 
%, to perform dynamic tree operations. \tcb{PERHAPS explain the above table widely or paraphrasing ???}

In the next section, we give an overview of the finger tree functions and their performance. These are the foundations for our functions altogether with some of the functions from the Hackage library \texttt{Data.Set} \footnote{https://hackage.haskell.org/package/containers-0.5.10.2/docs/Data-Set.html} which mimic the set-like operations over our functions, such as testing membership. 

%In Section~\ref{sec:Example}, we present an in-depth example using the operations mentioned above. In particular, we depict how the \dyntset manages the trees as sequences under the dynamic setting, showing how the \dyntset structure is constructed immutably. 
In Section \ref{sec:TechDes}, we describe our implementation for performing dynamic tree operations with purely data structures, the \dyntset. The following is the address where our source code is hosted and publicly available:
\begin{center}
\url{https://github.com/jcsaenzcarrasco/ETdynTs}
\end{center}

We evaluate \dyntset empirically in Section~\ref{sec:Eval}. In concrete, we plot the results of benchmarking the functions in the first two sections of the table above. Our evaluation demonstrates that \dyntset performs according the function definitions with a constant factor between \conn and \link and \cut. 

%Section~\ref{sec:Discuss} discuss the main reasons we took into account when designing ETFT, with some suggestions for improving the internal vertex annotations. 
In Section~\ref{sec:RelWrk} we suggest potential alternatives for designing dynamic data structures to solve the problem described in the present document. 

Finally, we conclude in Section~\ref{sec:Concl} with a summary of our work.


\tcr{Presentationally I'm missing a better description of the background material. The paper recalls the definition of a monoid, but does not recall the definition of an Euler tour nor give enough detail on finger trees to read as an independent document.} 


\section{Preliminaries} 
\label{sec:Prelim} 

Our work and contribution relies mainly on three structures, a \textit{monoid}, a \textit{set} and a \textit{finger tree}, which are briefly described in the following subsections. 

\subsection{Monoid}

A \textit{monoid} is a triple $(S,\star,e)$, where $S$ is a set, $\star$ is a binary operation, called $product$ and $e$ is an element of $S$, called $unit$, satisfying the following properties:

\begin{enumerate}
\item $e \star x = x = x \star e$, for all $x \in S$ 
\item $x \star (y \star z) = (x \star y) \star z $, for all $x,y,z \in S$.  
\end{enumerate}

As an example of a Haskell implementation of a monoid we have:
%\begin{verbatim}
\begin{lstlisting}[mathescape]
class Monoid  a where 
   mempty  :: a
   mappend :: a $\to$ a $\to$ a
\end{lstlisting}   
%\end{verbatim}

where the function \code{mempty} represents the element $e$ and the function \code{mappend} represents function $\star$. Detailed information about monoids within the functional programming can be found in \cite{Monoids}, and for the Haskell implementation at \cite{HaskellMonoid}.


\subsection{Set as binary search tree}
A \textit{binary search} tree (BST) is either a \textit{leaf} (also called a \code{tip}) or a vertex consisting of a \textit{value}, a \textit{left} BST and a \textit{right} BST. The height of the tree determines the time taken to perform every operation onto it, therefore the shorter the height the better, and this is done throughout a \textit{balancing scheme}. The study of BSTs is vast and there are several implementation around. In our case, a \textit{size balanced} is used. A detailed comparison and benchmarking is outside the scope of the present work.

\tcr{My main criticism of the paper is the presentation. While I appreciate the use of diagrams in the first half of the paper, high-level explanations are missing later. The paper does not do a very good job providing an intuition for how the data structure works. For example, I was struggling understanding the role of the monoidal set. The paper also explains too little of the background regarding used existing data structures and their realisation in Haskell libraries. For example, the interface to the finger tree ADT and the Measured type class. In general, most of the explanations are closely tight to the code but don’t give a high-level picture of what individual operations do in terms of the Euler tour. Given that there is plenty of space left in the paper, this could be improved in the final version.}


The following is an snippet of \code{Data.Set} \footnote{https://hackage.haskell.org/package/containers-0.5.10.2/docs/Data-Set.html}
%\begin{verbatim}
\begin{lstlisting}
data Set a    = Bin Size a (Set a) (Set a)
              | Tip
\end{lstlisting}              
%\end{verbatim}

In Table~\ref{tab:Setfuncs} we show the set-functions we have incorporated in \code{dynTsET} from \code{Data.Set}.
\small
\begin{table}
\begin{center}
\begin{tabular}{||l | l | c||} 
 \hline
 Function         & Type                                   & Time complexity            \\ 
 \hline\hline
 \texttt{empty}   & \texttt{Set a}                         & $O(1)$                     \\ 
 \hline
 \texttt{insert}  & \texttt{a $\to$ Set a $\to$ Set a}     & $O(\log n)$                \\
 \hline
 \texttt{member}  & \texttt{a $\to$ Set a $\to$ Bool}      & $O(\log n)$                \\ 
 \hline
 \texttt{union}   & \texttt{Set a $\to$ Set a $\to$ Set a} & $O(m(\log\frac{n}{m} +1))$ \\
 \hline
\end{tabular}
\caption{Leijen's implementation of \code{Data.Set} \cite{HaskellSet}, based on \cite{ParallelSets}}
\label{tab:Setfuncs} 
\end{center}
\end{table}
\normalsize

Functions \code{empty} and \code{union} are, in fact, the monoidal functions \code{mempty} and \code{mappend} respectively. 

\subsection{Euler-tour tree} 

Dealing with trees of different degree can be complicated. A simple way to handle and represent trees of any degree is by an Euler tour, that is, a sequence as in \cite{Rand-DynGs-Algos} and \cite{WerneckR-PhD}. To represent a tree $t$, we replace every edge $\langle u,v \rangle$ of $t$ by two arcs $(u,v)$ and $(v,u)$, and add a loop $(v,v)$ to represent each vertex $v$. In this context, a tree $t$ can have at least one, and in general, many Euler tours. The size of an Euler tour \textit{et} of $t$ is $et(t) = v + 2e $, where $v$ is the number of vertices of $t$ and $e$ its number of edges. We can represent an Euler tour in Haskell simply as a list of pairs, such in 

%\begin{verbatim}
\begin{lstlisting}
data EulerTour a = [(a,a)] 
\end{lstlisting}
%\end{verbatim}

By managing the tour with lists, we can perform insertion from the left (head) in $O(1)$ but remaining operations such insertion from the right, access, cutting, appending and inserting might take $O(n)$ per operation.
On the other hand, representing tours through finger trees, performance per operation is improved up to $O(\log n)$ per operation amortised as we explain shortly and in Section~\ref{sec:TechDes}. 

Our representation of $k$-trees through Euler-tour will not close up the tour with the first node as this avoids the uniqueness presence for such a node, as shown in Fig.~\ref{fig:Euler-tour}.

\begin{figure}
\begin{center}
\includegraphics[scale=0.35]{./Images/Euler-tour} 
\end{center}
\caption{The $k$-tree (left) is represented as a sequence of size $v+2e$ (right). Notice we left out the final node-pair to preserve uniqueness}
\label{fig:Euler-tour}
\end{figure}



\subsection{Finger tree} 

We present Hinze and Paterson's version of finger trees \madd{(FTs)} \cite{FTs}, followed by the functions we have used for the \code{dynTsET}.
%\begin{verbatim}
\begin{lstlisting}
data FingerTree v a = Empty
                    | Single a 
                    | Deep v 
                           Digit a 
                           FingerTree v (Node v a) 
                           Digit a
\end{lstlisting}                           
%\end{verbatim}


\tcr{This reviewer can't help but think that there seems to be a lot of redundancy in the proposed data structure, as the Euler tour node pairs are present in both the sets and in the finger trees.}


\code{Digit} type holds from one up to four elements of type \code{a}. \code{Node} type can hold two or three elements of type \code{a}. The recursive and nested definition of \code{FingerTree} forces the structure to be balanced by its types, instead of enforcing it by code invariants. Finger trees are 2-3 trees where values are stored at the leaves, in our case those leaves are the pairs representing an Euler tour. To implement \textit{updates} and \textit{lookups} efficiently, Hinze and Paterson \cite{FTs} added a monoidal annotation on the intermediate vertices, the \code {v} type.

In figure Fig.~\ref{fig:FT-Euler-tour} we can see our example of the Euler-tour sequence in Fig.~\ref{fig:Euler-tour} managed by the data constructors described above.

\begin{figure}
\begin{center}
\includegraphics[scale=0.35]{./Images/FT-Euler-tour} 
\end{center}
\caption{A finger tree (FT) holding the sequence (Euler-tour) that represents the $k$-tree in Fig.\ref{fig:Euler-tour}}
\label{fig:FT-Euler-tour}
\end{figure}




\tcb{SPLIT is not longer used, instead there is SEARCHFT, which aim is the same as SPLIT and hope it will be easier to explain}

\tcb{Is it the ``best" part in the document to bring up an explanations or examples or both for amortisation?}
%\section{Example}
\label{sec:Example} 

In this section, we provide an example of the ETFT interface by depicting the core structure (i.e. finger tree) with internal vertices (i.e. sets) and sequences of pairs on the leaves (i.e. tours). Also, we illustrate the steps for computing \textit{split}s, \textit{insertion}s, \textit{concatenation}s, and lookups or \textit{view}s. All these the supporting functions to \textit{link} and \textit{cut} operations. Later, in Section~\ref{sec:TechDes}, we define these functions precisely describing their performance. 

\tcr{The whole section 3 is clumsy. Some of the pictures are useful to explain the idea of the annotations, but in many cases neither the text nor the figure add much information to what is already obviously from the name and type signature.}



Let us define two trees, $t_8$ and $t_2$. In this case we will consider the top vertex as the root, although, this is arbitrary and we can \textit{reroot} the tree at an existent vertex at any time.

Firstly, we have the tree $t_8$ on the left in in Figure~\ref{fig:t1et1ft1}, followed by its (only) Euler tour representation $et_8$ and its finger tree $ft_8$. At the top of $ft_8$ we have the monoidal representation of the set-union operation (shadowed ellipse) over all the leaf-vertices containing the pairs from $et_8$.
\begin{figure}
\begin{center}
\includegraphics[scale=0.35]{./Images/t1et1ft1} 
\end{center}
\caption{Tree $t_8$ and its corresponding Euler tour $et_8$ and finger tree $ft_8$}
\label{fig:t1et1ft1}
\end{figure}

As an second example, we have $t_2$, Figure~\ref{fig:t2et2ft2}. The spine of the finger tree now contains a \code{Single} type constructor altogether with a \code{Node3}. Again, shadowed ellipses refer to the monoidal annotations \code{Set} where the set-insertion and set-union take place \textit{automatically}.
\begin{figure}
\begin{center}
\includegraphics[scale=0.35]{./Images/t2et2ft2} 
\end{center}
\caption{Tree $t_2$ and its corresponding Euler tour $et_2$ and finger tree $ft_2$}
\label{fig:t2et2ft2}
\end{figure}

Now, the forest $f$ in Figure~\ref{fig:forest} is formed by inserting $ft_8$ and $ft_2$ from left and right respectively, in $O(1)$. The forest's \code{Deep} type constructor has the set-union of all the finger trees ($ft_8$ and $ft_2$), depicted in back colour with white font. The thick lines coming out from the top ellipse are the components of the top finger tree: two \code{Digit} and an \code{Empty} subtree (or subforest).
\begin{figure}
\begin{center}
\includegraphics[scale=0.35]{./Images/forest} 
\end{center}
\caption{The forest $f$ comprising finger trees (Euler tours) as elements}
\label{fig:forest}
\end{figure}

Following, we have the basic operations for dynamic trees. Left side of Figure~\ref{fig:rootReroot} shows the \code{root} of a tree $t$ returning a vertex $v$, which is the first element of its finger tree $ft$, implemented in \code{viewl}. Rerooting a tree, by calling \code{reroot}, ask for a vertex $v$ and returns a \textit{new} tree. This is depicted in right side of Figure~\ref{fig:rootReroot}. This operation involves split, concatenation and insertions detailed in Section~\ref{sec:TechDes}.
\begin{figure}
\begin{center}
\includegraphics[scale=0.3]{./Images/rootReroot} 
\end{center}
\caption{\textit{root} and \textit{reroot} of a tree}
\label{fig:rootReroot}
\end{figure}

Splitting a finger tree, \code{split}, returns two subtrees (i.e. left and right) and an element, either a pair or an Euler tour, depending on the top container, a tree or a forest. This is depicted in Figure~\ref{fig:split}.
\begin{figure}
\begin{center}
\includegraphics[scale=0.25]{./Images/split} 
\end{center}
\caption{Splitting a tree returns two smaller trees altogether an element}
\label{fig:split}
\end{figure}

Concatenating (also called appending) takes two finger trees and returns another (bigger) finger tree, as in Figure~\ref{fig:concatenation} and its represented by the $\bowtie$ symbol.
\begin{figure}
\begin{center}
\includegraphics[scale=0.25]{./Images/concatenation} 
\end{center}
\caption{Concatenating two trees, returns a new one bigger}
\label{fig:concatenation}
\end{figure}


\tcr{Since the split operation over finger trees is central to the code, the paper would benefit from a more thorough explanation than the current incomplete type signatures (Concretely I'm missing an explanation of split's two first arguments).}


\tcr{The paper heavily builds on finger trees and tries to explain what they are and how they work, but omit aspects that are crucial to the understanding of the paper. In particular the `split` function is not explained, only listed in Table 2, but used in almost every listing later in the paper. The type listed for `split` mentions an unexplained type constructor `Split` and its type does not match the uses of `split`. The explanation around Fig. 5 does not help. What are the two first arguments to `split`?}




Under Figure~\ref{fig:consleftsnocright} we depict the final four help-functions that support \textit{link} and \textit{cut}: \code{viewl}, $\lhd$ (or \textit{cons}), $\rhd$ (or \textit{snoc}) and \textit{viewr}.
\begin{figure}
\begin{center}
\includegraphics[scale=0.25]{./Images/consleftsnocright} 
\end{center}
\caption{Inserting and viewing from left (\code{viewl},$\lhd$) and inserting and viewing from right (\code{viewr},$\rhd$) }
\label{fig:consleftsnocright}
\end{figure}







\section{dynTsET Design}
\label{sec:TechDes}  

We do not assume sequences representing Euler-tours are provided. Instead, we offer auxiliary functions that turn any-degree tree (and forest) into well-formed sequences. In Haskell, any-degree tree are also known as multiway or rose tree. Actually, there is a library that defines such trees, \code{Data.Tree}, in the Haskell community's central package archive (Hackage).

\begin{lstlisting}
data Tree a = Node {
        rootLabel :: a,         
        subForest :: Forest a   
    }
\end{lstlisting}

So, a tree is a node of type \code{a} (i.e. its root) altogether with a list (possibly empty) of subtrees, seen as forest of type \code{Forest a}. Analogous to multiway trees, our trees (finger trees holding Euler-tours) can not be empty. 

Firstly, we turn a tree into a list of nodes, by attaching (mapping) the prefix and suffix of current node to inner subtrees. We later feed this list into a finger tree. 

\begin{lstlisting}[mathescape] 
rt2et :: (Eq a) $\Rightarrow$ Tree a $\to$ [(a,a)] 
rt2et (Node x ts) = case ts of
  [] $\to$ [(x,x)]
  _  $\to$ root ++ concat ( map ($\lambda$t $\to$ pref t ++ rt2et t ++ suff t) ts )   
    where
     pref v = [(x,rootLabel v)]
     suff v = [(rootLabel v,x)]
     root   = [(x,x)] 
\end{lstlisting} 

Since \code{concat} and \code{++} take $O(n)$ where $n$ is the size of their arguments, \code{rt2et} takes $O(n^2)$. \tcr{Should we need to prove this ???}

The corresponding forest transformation to list of pairs is simply the application of \code{rt2et} to every element in the tree list. 

\begin{lstlisting}[mathescape] 
rf2et :: (Eq a) $\Rightarrow$ Forest a $\to$ [[(a,a)]]
rf2et []          = []
rf2et [Node x []] = [[(x,x)]]  
rf2et (t:ts)      = (rt2et t) : rf2et ts
\end{lstlisting}

Once a tree $t$ of any degree is turned into an Euler tour $et$, we are able to manage $et$ as a sequence. Since finger trees work efficiently on sequences, according to \cite{FTs}, we tailored the finger tree in order to support \link, \cut and \connected operations over unrooted trees following the corresponding properties and algorithm by Henzinger and King in \cite{Rand-DynGs-Algos}.

Recall data type in Fig~\ref{fig:FTdatatype} works with finger trees of any type \code{a} constrained to the monoidal annotation of type \code{v}. In order to manipulate nodes and edges as Euler tours, we specialized the \dyntset types to handle  pairs, hence the underlying finger tree types are \code{Set (a,a)} for the monoidal annotation (previously \code{v}) and \code{(a,a)} for values on the leaves (previously \code{a}). 

The full implementation of dynTsET\footnote{\url{https://github.com/jcsaenzcarrasco/ETdynTs}} contains the core functions for \code{link}ing, \code{cut}ting and asking for connectividad (\code{conn}) as well as the helper functions detailing the conversion between multiway trees and Euler-tours.

\tcb{SUBSECTION FOREST. Either here or before section 4.1}

For practical purposes we let forest and trees to be of type \code{Int} in this paper, but a polymorphic type can be extended easily.

We start with type definitions for nodes, trees, forests, and their empty instances as in : % Figure~\ref{fig:emptys}
\begin{lstlisting}[mathescape]
type TreeEF   a = FingerTree (S.Set (a,a)) (a,a)
type ForestEF a = FingerTree (S.Set (a,a)) (TreeEF a) 

emptyForest :: Ord a => ForestEF a  
emptyForest  = FingerTree.empty 

emptyTree :: Ord a => TreeEF a 
emptyTree  = FingerTree.empty 

instance (Ord a) => Measured (S.Set (a,a)) (a,a) where 
   measure (x,y) = S.insert (x, y) S.empty 
\end{lstlisting}   
%\caption{code for \code{tree},\code{forest},\code{vertex} definitions and their empty cases}
%\label{fig:emptys}
%\end{figure}

Every element in the tree $t$ (leaves in the finger tree) is measurable by calling \code{measure}, which returns the set-insertion operation of the element $(v,u)$ into an empty set. Here \code{Data.Set} is imported qualified as \code{S}

The \textit{root} of a tree $t$ returns a vertex $v$ or nothing if $t$ is empty:
%\small
%\begin{verbatim}
\begin{lstlisting}[mathescape]
root :: FTInt $\to$ Maybe Vertex  
root tree = case viewl tree of
  NoView   $\to$ Nothing
  View x _ $\to$ Just $\$$ fst x
\end{lstlisting}
%\end{verbatim}
%\normalsize

Time complexity of \code{root} is $O(1)$ since \code{viewl} performs $O(1)$ and we simply pattern matching over its results.

Then, \textit{rerooting} a tree $t$ according to vertex $v$ returns the same tree $t$ on which $v$ is the root, i.e. the first element in the sequence. This is useful when linking and cutting trees. Rerooting an empty or single tree is trivial (lines 3 and 4, Figure~\ref{fig:reroot}). 
\begin{figure}
%\begin{verbatim}
\begin{lstlisting}[mathescape]
reroot :: FTInt $\to$ Vertex $\to$ FTInt
reroot tree vertex = case tree of 
 Empty       $\to$ Empty 
 (Single x)  $\to$ Single x
 tree        $\to$ case split (S.member root) S.empty tree of 
                 NoSplit           $\to$ tree
                 Split Empty _  _  $\to$ tree 
                 Split tl    _  tr $\to$ let (View _ tX) = viewl tl 
                                          tA          = tX $\rhd$ root 
                                          tB          = root $\lhd$ tr 
                                      in  tB $\bowtie$ tA 
 where root = (vertex,vertex) 
\end{lstlisting} 
%\end{verbatim} 
\caption{code for \code{reroot} function}
\label{fig:reroot}
\end{figure}
Lines 6 and 7 show the cases that $v$ is not in $t$ or is already the root, respectively in Figure~\ref{fig:reroot}. From the left subtree of the split (line 8), we simply ignore the first element derived from its \code{viewl} which is the old root, then we simply concatenate the tail and the head in that order altogether the new root inserted in both sides. That is, one \textit{split}, one \textit{concatenation} and two insertions \textit{cons,snoc}, turns to be $O(\log n)$ in the size of the tree $t$.  


\subsection{connected} 
It is the first of our dynamic tree operations and the core function for \textit{link} and \textit{cut}. It receives two vertices $u$ and $v$ and a forest $f$ as arguments, returning a boolean altogether with a tree or trees and their roots where applicable. It is based on a single \textit{split} per vertex. If the split succeeds, then a tree and its root are returned, otherwise nothing is returned. This initial step is called \code{search} and takes as much as $O(\log n)$ per split and $O(1)$ for the \textit{root}. Then \code{connected} pattern matches all the cases from \code{search} and compares whether or not the given vertices are in the same tree. 

\begin{figure}
%\small
%\begin{verbatim}
\begin{lstlisting}[mathescape] 
search :: Vertex $\to$ Forest $\to$ Maybe (FTInt, Vertex) 
search v f = 
 case split (S.member (v,v)) S.empty f of 
  NoSplit        $\to$ Nothing 
  Split _ tree _ $\to$ Just (tree, fromJust $\$$ root tree) 

connected :: Vertex $\to$ Vertex $\to$ Forest $\to$ (Bool, Maybe PairTreeVertex) 
connected x y f = 
 case (search x f, search y f) of 
  (Nothing     , _           ) $\to$ (False, Nothing) 
  (_           , Nothing     ) $\to$ (False, Nothing) 
  (Just (tx,rx), Just (ty,ry)) $\to$ if rx == ry 
                                   then (True,  Just(tx,rx,tx,rx))  
                                   else (False, Just(tx,rx,ty,ry))  
\end{lstlisting} 
%\end{verbatim}
\caption{the \textit{connected} function, the basic query for dynamic trees and the core for \textit{link} and \textit{cut}}
\label{fig:connected}
\end{figure}
Following lines 9-14 in Figure~\ref{fig:connected}, the first two cases of \code{connected} occur when any of $u$ or $v$ are not members of $f$. Last two lines handle the restrictions for functions \textit{cut} and \textit{link} respectively. That is, if two vertices $u$ and $v$ are connected, then both have the same root of the same tree, preserving the \textit{cut} condition. The last line states the valid case for \textit{link}, that is, not connected $u$ and $v$ that belong to $f$, therefore \code{connected} also returns the corresponding trees and roots. Finally, we conclude that \code{connected} takes as much effort as \code{search}, $O(\log n)$, where $n$ is the number of vertices in the forest.

\subsection{link}
Performing a \textit{link} implies that given two vertices $u$ and $v$ and a forest $f$, $u$ and $v$ should belong to a different trees $t_u$ and $t_v$ in $f$. This condition is guarded by \code{connected}, in case of non existent $u$ or $v$ and no connection between, $f$ is simply returned. This will be useful when \code{link} is part of the unbound sequence of operations applied to a forest. The \textit{link} condition is stated and satisfied in the case analysis, lines 3-5 in Figure~\ref{fig:link}. 
Now, we need to reduce the forest $f$ by one tree as $t_u$ and $t_v$ will be joined. Hence $t_u$ and $t_v$ are extracted from $f$ and dropped off (left apart with \code{_}), lines 7 and 8 in Figure~\ref{fig:link}. Finally, \code{linkAll} joins subforests \code{lf,rf} and inserts the new tree (\code{tree}) resulting from auxiliary function \code{linkTree}.  

\begin{figure}
%\small
%\begin{verbatim}
\begin{lstlisting}[mathescape] 
link :: Vertex $\to$ Vertex $\to$ Forest $\to$ Forest 
link x y f = 
 case connected x y f of 
  (False, Just (tx,rx,ty,ry)) $\to$ linkAll (linkTree x tx y ty) 
  _                           $\to$ f 
 where 
    Split lf' _ rf' = split (S.member (x,x)) S.empty f 
    Split lf  _ rf  = split (S.member (y,y)) S.empty (lf' $\bowtie$ rf') 
    linkAll tree    = tree $\lhd$ (lf $\bowtie$ rf) 
\end{lstlisting} 
%\end{verbatim}
\caption{the \textit{link} function, reduces the size of the forest by one tree}
\label{fig:link}
\end{figure}

Inserting the new tree into the new forest takes $O(1)$ due \code{cons}, but it is the computation of \code{linkTree} that takes $O(\log m)$, where $m$ is the size of the trees $t_u$ and $t_v$ as it applies two \code{reroot} and a concatenation, lines 4-6 in Figure~\ref{fig:linkTree}. Finally, joining the subforests derived from eliminating trees $t_u$ and $t_v$ takes three splits, lines 3, 7 and 8, which take $O(\log n)$ each and a concatenation, line 9 which also takes $O(\log n)$, where $n$ is the number of vertices in the forest $f$. In the worst case, if $f$ is formed by a single tree $t$ after linking, then $n$ and $m$ are equivalent.

\begin{figure}
%\small
%\begin{verbatim}
\begin{lstlisting}[mathescape] 
linkTree :: Vertex $\to$ FTInt $\to$ Vertex $\to$ FTInt $\to$ FTInt 
linkTree u tu v tv =  
   let 
      from = (reroot tu u ) $\rhd$ (u,v)
      to   = (reroot tv v ) $\rhd$ (v,u) 
   in (from $\bowtie$ to ) $\rhd$ (u,u)  
\end{lstlisting} 
%\end{verbatim}
\caption{Auxiliar \textit{linkTree} function, the \textit{join} between two different trees}
\label{fig:linkTree}
\end{figure}



\subsection{cut}
Unlike \textit{link}, the function \textit{cut} adds a new tree into the forest $f$. The condition for \textit{cut} to be valid, is that input vertices $u$ and $v$ in forest $f$ should also belong the same tree $t$. We also add the condition that a vertex can not be cut, just edges. The latter condition is guarded in line 3 in Figure~\ref{fig:cut} whereas former condition is preserved by pattern matching in lines 6 and 7 within the same figure. Like \textit{link}, function \textit{cut} will return the original forest $f$ if any of these conditions are not satisfied. Again, this helps the unbounded sequence of dynamic operations to be applied over $f$.


\tcr{I miss a deeper discussion of the interplay of this use of the finger tree and the monoidal annotations. For example, the `reroot` operation should leave the set of edges invariant. Do the annotation on the top node have to be recalculated? Can it be avoided? Would it make a difference? }


\begin{figure}
%\small
%\begin{verbatim}
\begin{lstlisting}[mathescape] 
cut :: Vertex $\to$ Vertex $\to$ Forest $\to$ Forest 
cut x y f  
 | x == y    = f  
 | otherwise = 
    case connected x y f of 
      (True, Just (tx,_,_,_)) $\to$ buildForest (cutTree x y tx) 
      _                       $\to$ f 
 where 
    buildForest (t2,t3) = t2 $\lhd$ (t3 $\lhd$ (lf $\bowtie$ rf)) 
    Split lf _ rf       = split (S.member (x,x)) S.empty f 
\end{lstlisting} 
%\end{verbatim}
\caption{The \textit{cut} function increases the size of the forest $f$ by one tree}
\label{fig:cut}
\end{figure}

If the \textit{cut} function is successful, then tree $t$ is splitted into trees $t_u$ and $t_v$ according the input vertices $u$ and $v$, hence $t_u$ is the tree containing vertex $u$ but not $v$ and $t_v$ is the tree containing vertex $v$ but not $u$, denoted as trees \code{t2} and \code{t3} in Figure~\ref{fig:cut} respectively. Also, we need to remove $t$ from forest $f$. The insertion of $t_u$ and $t_v$ into $f$ and removal $t$ from $f$ is done in lines 9 and 10. Thus, function \textit{cut} takes two splits, one from \code{connected} in line 5, and from line 10, which turns in $O(\log n)$. Also, a concatenation is performed in rebuilding the forest $f$, in line 9, adding up another $O(\log n)$, where $n$ is the number of vertices in $f$. Additionally to this, auxiliary function \code{cutTree}, pictured in Figure~\ref{fig:cutTree}, takes one viewing (\code{viewr}) which is $O(1)$, then, one rerooting, two splits and one concatenation, each of $O(\log m)$, where $m$ is the number of elements in $t$. In the worst case, if $f$ contains a single tree $t$ before cutting, then $m$ and $n$ are equivalent.

\begin{figure}
%\small
%\begin{verbatim}
\begin{lstlisting}[mathescape] 
cutTree :: Vertex $\to$ Vertex $\to$ FTInt $\to$ (FTInt,FTInt) 
cutTree u v tree = case split (S.member (u,v)) S.empty tree of
 NoSplit $\to$ (tree,Empty) 
 _       $\to$ 
  let treeU          = reroot tree u 
      Split  treeA' _ right = split (S.member (u,v)) S.empty treeU
      View _ treeA          = viewr treeA' 
      Split treeB _ treeC   = split (S.member (v,u)) S.empty right 
  in  (treeB, treeA $\bowtie$ treeC) 
\end{lstlisting} 
%\end{verbatim}
\caption{Auxiliary function \textit{cutTree} \textit{splits} the given tree in two}
\label{fig:cutTree}
\end{figure}



 
\section{Experimental Evaluation} 
\label{sec:Eval} 


This section presents experiments to evaluate how much running time costs in terms of performance. The experiments will show that, in practice, \dyntset is faster by a factor of $O(\log n)$ per operation than that of the theoretical analysis (Section\ref{sec:TechDes}).

This section is organised as follows. Firstly, we describe the experimental setup. Secondly, a brief description in the implementation of test sets is provided. We then present experimental studies of the three different operations in \dyntset. Finally, we present an additional experiment for the cases where laziness as speeding up factor in favor of the running times for the dynamic tree operations.


\subsection{Experimental Setup}
Functions \link, \cut, \conn, \code{root} and \code{reroot} were implemented by the author in Haskell and compiled with \code{ghc} version 8.0.1 with optimisation \code{-O2}. The experiments were performed on a 2.2 GHz Intel Core i7 MacBook Pro with 16 GB 1600 MHz DDR3 running macOS High Sierra version 10.13.1 (17B1003). We imported the following libraries into our code from the online package repository Hackage: \cite{HaskellFT} code for finger trees, \cite{HaskellSet} for conventional sets and \cite{HaskellEdison} for lazy sets.

The running time of a given computation was determined by the mean of three executions.

\subsection{Data structure} 

The values maintained by the data structures (sets and finger trees) are stored as fixed-precision \code{Int} types, holding values from $-2^{29}$ up to $2^{28}$ although we test only the positive values.

The structures are initialized with a fixed number of nodes (or vertices) $n$; this number does not change during the execution. This allow us to know the initial size of the forest and we subtract it from the benchmarking.

Since \dyntset is not called by any application, the random generation of nodes for \link or \cut does not necessarily be effective. Actually, around 70\% of the generated nodes $x$ and $y$ passed to \link and \cut were not valid, that is, their result turned out to be the original forest. In order to overcome this, we stored the random generated nodes that were effective into a plain files and from there benchmarking the dynamic tree operations.


\subsection{Incremental operations} 
We start with an empty forest (just singleton-trees); given $n=20,000$ nodes we perform $1 \ldots 20,000$ \link operations. Upon reaching a target length, we plot the total time taken. Then, we divide the time taken by the number of operations to calculate the time per operation and then multiply it by a constant (x1000) to make the curve visible in the same chart.

\begin{figure}[H]
\begin{center}
\includegraphics[scale=0.4]{./Images/plotLink} 
\end{center}
\caption{Sequence of {\link}s from empty forest up to a single tree in such forest}
\label{fig:incLink}
\end{figure}

\textit{\emph{Results}}. The behaviour of the curve regarding the time per \link operation shows that in practice it takes $O(1)$ against $O(\log n^2)$ in theory back in Section~\ref{sec:TechDes}, or the linear behaviour by the \link operations in bulk.

\subsection{Fully dynamic operations} 
We start with the incremental process as before for $n=10,000$. Then, for \cut we start in the opposite direction, that is, cutting from a single tree in the forest until only singleton-trees remain in such forest. To this performance we subtract the time take for the incremental bit. For \conn performance we compute first an interleaved operation of \link and \cut (not necessarily in this order). We measure the time taken for \conn followed by the corresponding \link or \cut and then we subtract the interleaved process. The following figures show our three dynamic operations in bulk and per operation.

\begin{figure}[H]
\centering
\begin{subfigure}{.5\textwidth}
  \centering
  \includegraphics[scale=0.38]{./Images/plotEach}
  \caption{In bulk}
%  \label{fig:sub1}
\end{subfigure}%
\begin{subfigure}{.5\textwidth}
  \centering
  \includegraphics[scale=0.38]{./Images/plotOpsIndiv}
  \caption{Per operation}
%  \label{fig:sub2}
\end{subfigure}
\caption{Time taken by operation, and interleaved \link and \cut}
\label{fig:EachOp}
\end{figure}

\textit{\emph{Results}}. We observe that \cut and \conn obey the same pattern as \link. That is, $O(1)$ time per operation being \textit{connectivity} the fastest of the dynamic tree operations, as expected.

From the above analyses, we notice that \link performs better when is interleaved with \cut. To see this behaviour closer, we present the bulk and individual cases in the following charts varying the forest size under the same amount of operations.

\begin{figure}[H]
\centering
\begin{subfigure}{.5\textwidth}
  \centering
  \includegraphics[scale=0.38]{./Images/plotForests}
  \caption{In bulk}
%  \label{fig:sub1}
\end{subfigure}%
\begin{subfigure}{.5\textwidth}
  \centering
  \includegraphics[scale=0.38]{./Images/plotLCForests}
  \caption{Per operation}
%  \label{fig:sub2}
\end{subfigure}
\caption{Time taken when \link and \cut are interleaved with different forest sizes}
\label{fig:EachOp}
\end{figure}

\subsection{Selection of the set data structure}
The set-like data structure is crucial in our implementation and testing of \dyntset since is the search engine for the nodes when any operation is applied to a forest. There are plenty of implementations for such set-like structure, mostly as binary balanced search trees. In our case, where Haskell is a lazy-evaluation language by default, we select two main choices to compare: \code{Data.Set} which is a strict data type definition and \code{Data.Edison.Coll.LazyPairingHeap} which is semi-lazy or semi-strict data type. The following figure shows the performance for each.

\begin{figure}[H]
\begin{center}
\includegraphics[scale=0.4]{./Images/plotSets} 
\end{center}
\caption{Dynamic operations through different sets structures as monoidal annotations}
\label{fig:plotSets}
\end{figure}

The above curves show that, although by a constant factor, laziness speeds up the running time in the computation of dynamic tree operations through the set-like data structures.

\tcb{Amortised over what??}
 
\section{Related Work} 
\label{sec:RelWrk} 

Approaches different to our own have, of course, been taken by other researchers, in particular by Dexter et al. \cite{Lazy-Gproc-Haskell} as well as by Headley and Hammer \cite{RAZ}. The former uses techniques to delay the computation of updates and the latter uses lighter and simpler data structures. Considering our own work alongside these approaches leads us to suggest that it may be possible to produce an interesting combination of the three, since there are at least three aspects we have found to be experimentally time-consuming in practice:

\begin{enumerate}
\item an Euler tour is treated as a sequence by efficient structures like finger trees, but the corresponding tree-like operations are also involved in update operations. A mixed, or lighter structure is an essential step towards achieving general efficiency;

\item the unbounded sequence of update operations results in a need to maintain the whole structure. In this context persistence triggers the consumption of memory allocation, and this suggests that benefits may accrue if we were to use higher order programming with effects;

\item the internal nodes of our finger tree structure hold the internal monoid \code{Set}, which is the most time-consuming element in our code and experimental results. Since rebalancing is an essential property of the internal binary search tree in \code{Set}, we propose conducting an analysis and design of different new balanced binary search tree as monoid in future work.
\end{enumerate}
 
\section{Discussion and Future Work} 
\label{sec:Concl} 
\label{sec:discussion}

Although we have presented evidence that ETFT trees are efficient data structures for dynamic tree handling, we have noticed experimentally that there is plenty of work to do specifically in the stream of interleaved operations applied as input to the data structure, that is, the unbounded sequence of updates and queries.

Finally, there is an evident need for a library of well-crafted test cases against which implementations of dynamic trees and graphs can be tested. At present, for example, there is little to guide us when generating the update-sequences against which our structures are validated, and this raises a number of obvious issues. For example,

\begin{itemize}
\item Can we find test sets which guarantee coverage of key properties?

\item Which properties are we interested in?

\item If we know the general ratio of updates to queries (say), can we identify a more specific data structure to enhance efficiency still further?

\item Or maybe it’s enough to test solutions empirically, using live data from (say) the ever-changing structure of links in the Internet of Things?

\end{itemize}

These and other questions remain to be addressed, but we believe the quest for efficient dynamic data structures will become ever more important as seek to model, simplify and reason about the increasingly dynamic algorithmic structures in which modern culture is immersed and on which it depends.


\tcb{Uniqueness on edges allow to carry labels, therefore ETFT could be the core for other approaches to dynamic trees such link-cut trees. Examples, illustrations, references ?? } 

\tcb{Acknowledgements:  chahine.moreau@gmail.com ?? } 
\bibliography{./Refs/refs}

\end{document}
